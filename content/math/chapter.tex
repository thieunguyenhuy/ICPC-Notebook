\chapter{Mathematics}

\section{Math notes}
    \subsection{Derangements}
    Count the number of permutations of length $n$ with $0$ fixed points (i.e.\ $p(i) \neq i$ for all $i \leq n$)
    \[
    f(n) = 
    \begin{cases}
    1, & n = 0,\\[6pt]
    0, & n = 1,\\[6pt]
    (n - 1)\bigl(f(n - 1) + f(n - 2)\bigr), & n \ge 2.
    \end{cases}
    \]
    
    An equivalent identity is
    \[
    f(n) = n\,f(n-1) + (-1)^n.
    \]

    \subsection{Catalan Numbers}
        \begin{itemize}
        \item Correctly matched parentheses expressions of length $2n$,
        \item Rooted binary trees with $n+1$ leaves,
        \item Dyck paths of length $2n$,
        \item Ways to triangulate a polygon with $n+2$ sides,
        \item Different binary search trees that can be constructed with $n$ distinct keys.
        \end{itemize}
        \[
        C_n = \frac{1}{n+1} \binom{2n}{n}, \quad n \ge 0.
        \]
        \[
        C_n = \binom{2n}{n} - \binom{2n}{n+1}, \quad n \ge 0.
        \]
        \[
        C_0 = 1, \qquad 
        C_{n+1} = \sum_{i=0}^{n} C_i \, C_{n-i}, \quad n \ge 0.
        \]
        
        The first few values of the sequence $(C_n)$ are
        \[
        1,\, 1,\, 2,\, 5,\, 14,\, 42,\, 132,\, 429,\, 1430,\, 4862,\, \ldots
        \]
    \subsection{Binet's Formula}
        \[
        F_0 = 0, \quad F_1 = 1, \quad F_n = F_{n-1} + F_{n-2} \quad \text{for} \quad n \ge 2.
        \]
        
        \[
        F_n = \frac{\phi^n - \psi^n}{\sqrt{5}},
        \]
        where \(\phi = \frac{1 + \sqrt{5}}{2}\) is the golden ratio and \(\psi = \frac{1 - \sqrt{5}}{2}\) is its conjugate.
    \subsection{Sums}
        \[
        1^4 + 2^4 + 3^4 + \cdots + n^4 = \frac{n(n+1)(2n+1)(3n^2+3n-1)}{30}.
        \]
    \subsection{Sitrling number of the second kind}
    The Stirlin number of the second kind, \(S(n, k)\), counts the number of ways to partition a set of \(n\) objects into \(k\) non-empty, unlabeled subsets. The formula is recursive and is given by:  
    \[
    S(n, k) = k \cdot S(n-1, k) + S(n-1, k-1), \quad S(0, 0) = 1, \quad S(n, 0) = 0 \quad \text{for} \quad n > 0.
    \]
    \[
    S(n, k) = \frac{1}{k!} \sum_{i=0}^{k} (-1)^{k-i} \binom{k}{i} i^n.
    \]
    \subsection{Mobius}
    \begin{center}
        \includegraphics[width=0.4\linewidth]{a.png}
    \end{center}
    \begin{center}
        \includegraphics[width=0.4\linewidth]{b.png}
    \end{center}
    \begin{center}
        \includegraphics[width=1.0\linewidth]{c.png}
    \end{center}
    \begin{center}
        \includegraphics[width=0.6\linewidth]{d_1.png}
    \end{center}
    \subsection{Combinatorics}
    \[
    \sum_{k=0}^{x} \frac{\binom{x}{k}}{\binom{n}{k}} = \frac{n + 1}{n - x + 1}.
    \]

    \subsection{Useful Geometry Formulas}

    \subsubsection{Triangle Geometry}
    
    \paragraph{Heron's Formula.}
    For a triangle with side lengths $a, b, c$ and semiperimeter
    \[
    s = \frac{a+b+c}{2},
    \]
    the area is
    \[
    A = \sqrt{s(s-a)(s-b)(s-c)}.
    \]
    
    \paragraph{Area Formulas.}
    For a triangle with side lengths $a,b,c$, height $h_a$, circumradius $R$, and inradius $r$:
    \[
    A = \frac{1}{2} a h_a,
    \qquad
    A = rs,
    \qquad
    A = \frac{abc}{4R},
    \qquad
    A = \frac{1}{2} ab \sin \alpha.
    \]
    
    \textit{From these, the circumradius and inradius are:}
    \[
    R = \frac{abc}{4A},
    \qquad
    r = \frac{A}{s}.
    \]
    
    \paragraph{Law of Sines.}
    \[
    \frac{a}{\sin A} = \frac{b}{\sin B} = \frac{c}{\sin C} = 2R.
    \]
    
    \paragraph{Law of Cosines.}
    \[
    a^2 = b^2 + c^2 - 2bc \cos A,
    \quad
    b^2 = c^2 + a^2 - 2ca \cos B,
    \quad
    c^2 = a^2 + b^2 - 2ab \cos C.
    \]
    
    \subsubsection{Analytic Geometry}
    
    Let $P(x_0, y_0)$ be a point and a line:
    \[
    \ell: \quad ax + by + c = 0
    \]
    
    \paragraph{Perpendicular Foot from a Point to a Line.}
    \[
    H\left(
    x_0 - \frac{a(ax_0 + by_0 + c)}{a^2 + b^2},
    \quad
    y_0 - \frac{b(ax_0 + by_0 + c)}{a^2 + b^2}
    \right).
    \]
    
    \paragraph{Signed Distance from a Point to a Line.}
    \[
    d = \frac{ax_0 + by_0 + c}{\sqrt{a^2 + b^2}}.
    \]
    
    \paragraph{Reflection of a Point Across a Line.}
    \[
    P' \left(
    x_0 - \frac{2a(ax_0 + by_0 + c)}{a^2 + b^2},
    \quad
    y_0 - \frac{2b(ax_0 + by_0 + c)}{a^2 + b^2}
    \right).
    \]
    
    \paragraph{Signed Angle Between Two Vectors.}
    For vectors $\vec{u} = (u_x, u_y)$ and $\vec{v} = (v_x, v_y)$, the signed angle $\theta$ from $\vec{u}$ to $\vec{v}$ is
    \[
    \theta = \operatorname{atan2}\!\left(
    u_x v_y - u_y v_x,\;
    u_x v_x + u_y v_y
    \right).
    \]
    Equivalently,
    \[
    \tan \theta
    = \frac{\vec{u} \times \vec{v}}{\vec{u} \cdot \vec{v}}
    = \frac{u_x v_y - u_y v_x}{u_x v_x + u_y v_y}.
    \]

\section{Algorithms}
	\kactlimport{MillerRabin.h}
	\kactlimport{FastFourierTransform.h}
	\kactlimport{GaussElimination.h}
	\kactlimport{GaussElimination2.h}

